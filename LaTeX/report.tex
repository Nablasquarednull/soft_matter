\documentclass[twoside,twocolumn,9pt]{article}
\usepackage{extsizes}
\usepackage[super,sort&compress,comma]{natbib} 
\usepackage[version=3]{mhchem}
\usepackage[left=1.5cm, right=1.5cm, top=1.785cm, bottom=2.0cm]{geometry}
\usepackage{balance}
\usepackage{times,mathptmx}
\usepackage{sectsty}
\usepackage{graphicx} 
\usepackage{lastpage}
\usepackage[format=plain,justification=justified,singlelinecheck=false,font={stretch=1.125,small,sf},labelfont=bf,labelsep=space]{caption}
\usepackage{float}
\usepackage{fancyhdr}
\usepackage{fnpos}
\usepackage[english]{babel}
\addto{\captionsenglish}{%
	\renewcommand{\refname}{Notes and references}
}
\usepackage{array}
\usepackage{droidsans}
\usepackage{charter}
\usepackage[T1]{fontenc}
\usepackage[dvipsnames]{xcolor}
\usepackage{setspace}
\usepackage[compact]{titlesec}
\usepackage{hyperref}
\usepackage{multirow}
\usepackage{tikz}
\usepackage{algorithm}
\usepackage{algpseudocode}
%%%Please don't disable any packages in the preamble, as this may cause the template to display incorrectly.%%%

%\usepackage{epstopdf}%This line makes .eps figures into .pdf - please comment out if not required.

\definecolor{cream}{RGB}{222,217,201}

\begin{document}
	
	\pagestyle{fancy}
	\thispagestyle{plain}
	\fancypagestyle{plain}{
		
		%%%HEADER%%%
		\renewcommand{\headrulewidth}{0pt}
	}
	%%%END OF HEADER%%%
	
	%%%PAGE SETUP - Please do not change any commands within this section%%%
	\makeFNbottom
	\makeatletter
	\renewcommand\LARGE{\@setfontsize\LARGE{15pt}{17}}
	\renewcommand\Large{\@setfontsize\Large{12pt}{14}}
	\renewcommand\large{\@setfontsize\large{10pt}{12}}
	\renewcommand\footnotesize{\@setfontsize\footnotesize{7pt}{10}}
	\makeatother
	
	\renewcommand{\thefootnote}{\fnsymbol{footnote}}
	\renewcommand\footnoterule{\vspace*{1pt}% 
		\color{cream}\hrule width 3.5in height 0.4pt \color{black}\vspace*{5pt}} 
	\setcounter{secnumdepth}{5}
	
	\makeatletter 
	\renewcommand\@biblabel[1]{#1}            
	\renewcommand\@makefntext[1]% 
	{\noindent\makebox[0pt][r]{\@thefnmark\,}#1}
	\makeatother 
	\renewcommand{\figurename}{\small{Fig.}~}
	\sectionfont{\sffamily\Large}
	\subsectionfont{\normalsize}
	\subsubsectionfont{\bf}
	\setstretch{1.125} %In particular, please do not alter this line.
	\setlength{\skip\footins}{0.8cm}
	\setlength{\footnotesep}{0.25cm}
	\setlength{\jot}{10pt}
	\titlespacing*{\section}{0pt}{4pt}{4pt}
	\titlespacing*{\subsection}{0pt}{15pt}{1pt}
	%%%END OF PAGE SETUP%%%
	
	%%%FOOTER%%%
	\fancyfoot{}
	\fancyhead{}
	\renewcommand{\headrulewidth}{0pt} 
	\renewcommand{\footrulewidth}{0pt}
	\setlength{\arrayrulewidth}{1pt}
	\setlength{\columnsep}{6.5mm}
	\setlength\bibsep{1pt}
	%%%END OF FOOTER%%%
	
	%%%FIGURE SETUP - please do not change any commands within this section%%%
	\makeatletter 
	\newlength{\figrulesep} 
	\setlength{\figrulesep}{0.5\textfloatsep} 
	
	\newcommand{\topfigrule}{\vspace*{-1pt}% 
		\noindent{\color{cream}\rule[-\figrulesep]{\columnwidth}{1.5pt}} }
	
	\newcommand{\botfigrule}{\vspace*{-2pt}% 
		\noindent{\color{cream}\rule[\figrulesep]{\columnwidth}{1.5pt}} }
	
	\newcommand{\dblfigrule}{\vspace*{-1pt}% 
		\noindent{\color{cream}\rule[-\figrulesep]{\textwidth}{1.5pt}} }
	
	\makeatother
	%%%END OF FIGURE SETUP%%%
	
	%%%TITLE AND AUTHORS%%%
	\twocolumn[
	\vspace{-1cm}
	\begin{@twocolumnfalse}	
		International Master 1 Physics, Photonics, Nanotechnology / Quanteem \hfill Soft Matter 2025-2026
		\begin{center}
			\noindent\LARGE{\textbf{Numerical Simulations of Ideal Chain Model of Polymer\\using the Freely Jointed Chain (FJC)}}\\
			\vspace{.5cm}
			\noindent\large{Andres Alvarez \& Diego Guerrero} \\
		\end{center}\\	
	\end{@twocolumnfalse} \vspace{0.6cm}
	
	]
	%%%END OF TITLE, AUTHORS AND ABSTRACT%%%
	
	%%%FONT SETUP - please do not change any commands within this section
	\renewcommand*\rmdefault{bch}\normalfont\upshape
	\rmfamily
	\section*{}
	\vspace{-1cm}
	
	%%%MAIN TEXT%%%
	
	\section{Introduction}
Polymer physics deals with complex substances through a simple approach. Understanding the properties of polymers from a molecular point of view is key to unveil their complexity.

Given an ensamble of polymers, we can use tools from statistics to derive some properties that apply to the ensamble at any given instant. To do so we model each polymer as a (continuous or discrete) ideal chain of monomers along with some constraints. Some examples of these models include the Freely Jointed Chain, the Freely Rotating Chain, the wormlike chain, the hindered rotating chain, the rotational isometric state model, among others \cite{Rubinstein}.


The first, most straightforward (yet illustrating)  approach is the Freely Jointed Chain model. The FJC is a chain consisting of \(N\) links, each of length \(b\) and able to point in any direction independently of each other (thus the simplicity).

To characterize the spatial extent of the polymers, two standard measures of extent are introduced: the end-to-end vector $\mathbf{Q}$ and the radius of gyration $R_g$. The end-to-end vector is defined as $\mathbf{Q} = \mathbf{r}_N - \mathbf{r}_0$, and provides a direct measure of the global elongation of a polymer chain.The statistical quantity derived from it, $\langle Q^2 \rangle$, is directly related to experimental observables obtained from single-molecule techniques, including force--extension measurements using optical tweezers, as well as from scattering experiments through their connection to the structure factor. 

The radius of gyration $R_g$ quantifies the spatial distribution of monomers around the center of mass of the chain and is defined as the mean squared distance of the monomers from this center. It provides a global measure of the polymer size and compactness that is less sensitive to instantaneous end-to-end fluctuations than $\mathbf{Q}$. Experimentally, $R_g$ is commonly accessed via small-angle neutron or X-ray scattering (SANS/SAXS), where it determines the low-$k$ behavior of the scattering intensity. Taken together, $\mathbf{Q}$ and $R_g$ offer complementary descriptions of polymer extent, allowing for a consistent comparison between numerical simulations and experimental measurements.

The goal of this practical work is to analyze the conformational statistics of a freely jointed polymer chain using numerical simulations and to connect microscopic configurations with measurable macroscopic observables.
	\section{Methods}
\subsection{Analysis of FJC polymer configurations}
\label{subsec:analysisFJC}



  The first script analyzes the simulation data of a \(3D\) FJC polymer for different chain lenghts \(N\). 
  The parameters are the bond length \(b\), the \(N\) values we wish to test for, and the number of configurations \(T\) we wish to analyze.
  An XYZ file containing T polymer configurations is read, and for every configuration the end-to-end distance \(Q\) and the radius of gyration \(R_{g}\) is computed. Finally both quantities are squared and averaged over all configurations yielding \(\langle Q^2\rangle$ and $\langle R_g^2\rangle\). This algorithm is looped over for every \(N\).	

\begin{algorithm}[H]
\caption{Analysis of FJC Polymer Configurations: $\langle Q^2 \rangle$ and $\langle R_g^2 \rangle$}
\begin{algorithmic}[1]
\State Set bond length $b$ and list of chain lengths $\{N\}$
\State Set number of configurations $T$
\State Initialize arrays $\langle Q^2 \rangle_{\text{sim}}$ and $\langle R_g^2 \rangle_{\text{sim}}$

\For{each chain length $N$}
    \State Initialize position arrays $\{X,Y,Z\}$ of size $T \times (N+1)$
    \State Open trajectory file     
    \For{$t = 1$ to $T$}
        \For{$i = 1$ to $N+1$}
            \State Read monomer coordinates $(x_i,y_i,z_i)$
            \State Store coordinates in $X(t,i), Y(t,i), Z(t,i)$
        \EndFor
    \EndFor
    \State Initialize arrays $Q(t)$ and $R_g(t)$
    
    \For{$t = 1$ to $T$}
        \State Compute end-to-end distance
        \State Compute center of mass
        \State Compute radius of gyration
    \EndFor
    
    \State Compute ensemble averages
\EndFor
\end{algorithmic}
\end{algorithm}
\subsection{Statistical analysis of Q}
\label{subsec:Q_analysis}


The second script is built over the first one. The only extra parameter needed are the number of bins for the histogram. After computing the end-to-end distance \(Q\), the algorithm builds a histogram out of these values, treating them as samples from a probability distribution. The histogram is then normalized so that it can be directly compared to the gaussian distribution.


\begin{algorithm}[H]
\caption{Statistical analysis of Q}
\begin{algorithmic}[1]
\State Replicate Algorithm No.1
\State Set the number of bins
\State Construct normalized histogram of $Q$ values to estimate $P(Q)$
\State Evaluate theoretical distribution:
\[
P(Q) = 4\pi Q^2 \left( \frac{3}{2\pi N b^2} \right)^{3/2}
\exp\!\left( -\frac{3 Q^2}{2 N b^2} \right)
\]
\State Compare simulated and theoretical distributions
\end{algorithmic}
\end{algorithm}

\subsection{Structure Factor}
\label{subsec:structure}


The third simulation focuses on computing the structure factor for a polymer of \texttt{N = 100} and \texttt{b = 3}. An array \(k\) is introduced, spanning values from 0 to 1 (in our case, sampling every 0.01 steps). One of the files containing the simulated polymers is held in memory. Then, we use the formula  

\begin{equation}
I(k) = \sum_{i=0}^{N} \sum_{j=0}^{N}
\left\langle
\frac{\sin\!\left(k\,|\mathbf{R}_i - \mathbf{R}_j|\right)}
     {k\,|\mathbf{R}_i - \mathbf{R}_j|}
\right\rangle
\end{equation}
where \(\mathbf{R}_{i}\),\(\mathbf{R}_{i}\) are coordinates of the monomers in 3D space obtained from the simulated polymer.

Then, the Guinier approximation is computed on the same \(k\) array to compare them side by side.

\begin{algorithm}[H]
\caption{Structure Factor and Guinier approximation}
\begin{algorithmic}[1]
\State Set bond length $b$, chain length $N$, and number of configurations $T$
\State Define a range of scattering wave vectors $k$

\For{each value of $k$}
    \For{each pair of monomers $(i,j)$}
        \State Compute distance between monomers $i$ and $j$
        \If{distance is zero}
            \State Assign unit contribution
        \Else
            \State Assign sinc-like contribution depending on distance and $k$
        \EndIf
    \EndFor
\EndFor

\State Sum contributions over all monomer pairs to obtain the structure factor
\State Compute the radius of gyration from model parameters
\State Evaluate the Guinier approximation using the computed $R_g$

\State Compare numerical structure factor with the Guinier approximation in the low-$k$ regime
\end{algorithmic}
\end{algorithm}
 
\subsection{Polimer Extension}
A considerable ammount of attention has been driven toward the study of polymers that are not large to be on termodynamic equilibrium, the extension of a polymer by the stretching of a cantilever is a recurrent example \cite{Chen2010} in this regard, an expression for the force - extension relationship has been found by \cite{extensionRef}.  

\begin{equation}
  | \vec{Q} \cdot \vec{u}_x | = N b \left[ \coth{\alpha} - \frac{1}{\alpha}\right];
  \\
  \alpha = \frac{Fb}{K_B T}
  \label{eq:force-ext}
\end{equation}

In order to test this analytical formula with experimental data, a simulation was performed, where one polymer of 100 monomers was exposed to a range of constant forces from 0 to 10 pN. The polymer is modeled as a chain of fixed-length bonds, and its configuration is updated by randomly perturbing individual bond orientations. 

Each trial move changes the end-to-end vector and the associated potential energy due to the applied force. The acceptance of these moves follows a Boltzmann-like (Metropolis) criterion, which ensures that configurations are sampled according to thermal equilibrium at temperature 
T
T. This energy-based acceptance allows the algorithm to capture thermal fluctuations and generate physically meaningful polymer conformations under force. For the details of the algorithm look to Algorithm 1. 

\begin{algorithm}[H]
\caption{Monte Carlo Polymer Extension Under Force}
\begin{algorithmic}[1]
\State Initialize $N$ bond vectors uniformly on the unit sphere, scaled by bond length $b$
\State Compute monomer positions and end-to-end vector $\mathbf{Q}$
\State Set external force $\mathbf{F} = (F_x,0,0)$ and energy $U = -\mathbf{F}\cdot\mathbf{Q}$

\For{$\text{step} = 1$ to $n_{\text{steps}}$}
    \State Randomly select bond index $i$
    \State Propose new bond direction $\hat{u}$ and bond $\mathbf{b}_i' = b\hat{u}$
    \State Compute proposed $\mathbf{Q}'$ and $U'$
    \State $\Delta U \gets U' - U$
    \If{$\Delta U < 0$ \textbf{or} $\mathrm{rand} < e^{-\Delta U/k_BT}$}
        \State Accept move: update bonds, $\mathbf{Q} \gets \mathbf{Q}'$, $U \gets U'$
        \State Recompute monomer positions
    \EndIf
    \State Store $\mathbf{Q}$ and $U$
    \If{$\text{step} \bmod n_{\text{skip}} = 0$}
        \State Save polymer configuration to file
    \EndIf
\EndFor
\State Combine saved frames into a trajectory file
\end{algorithmic}
\end{algorithm}


\begin{figure}[ht]
		\label{figure1}
		\begin{minipage}{.45\linewidth}
			\centering
			\includegraphics[height=4cm]{figures/fig_1.png}
		\end{minipage}
		\hfill
		\begin{minipage}{.45\linewidth}
			\centering
			\includegraphics[height=4cm]{figures/fig_2.png}
		\end{minipage}		
		\caption{Two polymer structures generated from FJC model simulations for $N=100$ and $b=3.0$.}
	\end{figure}
	
	\section{Results and Discussion}
   \subsection{Analysis of FJC Polymer}
  \label{subsec:algo1}
  Figure \ref{fig:whatever} shows qualitatively the figures of measure approximated against the theoretical results. It is evident that the scaling of the parameter N does not influence the precision of the figures of measure, as they remain consistent with theory.
	\begin{figure}[ht]
		\begin{minipage}{.45\linewidth}
			\centering
			\includegraphics[height=4cm]{../code/Q2_vs_N_theory.pdf}
		\end{minipage}
		\hfill
		\begin{minipage}{.45\linewidth}
			\centering
			\includegraphics[height=4cm]{../code/Rg2_vs_N_theory.pdf}
		\end{minipage}		
		\caption{$\langle Q^{2}\rangle$ (left) and $\langle R_{g}^{2}\rangle$ (right)}
    \label{fig:whatever}
	\end{figure}
\subsection{Statistical analysis of Q}
\label{subsec:statistical_analisis}

In figure \ref{fig:histogram} it is seen that with a sufficiently high number of bins, the simulated disitribution ressembles the gaussian one. After a few trials, we opted for 50 bins in our algorithm, since it is sufficient to show the nature of the distribution.
  \begin{figure}[H]
      \begin{minipage}{0.95\linewidth}
      \centering
      \includegraphics[height=4cm]{../code/Histogram_Q.pdf}
      \end{minipage}
      \caption{Probability distribution}
      \label{fig:histogram}
  \end{figure}

  Figure \ref{fig:structure} shows the simulated structure factor for the selected \texttt{k} array, while figure \ref{fig:comparison} is a close-up at the values \texttt{k = [0.01,0.3]}. One may see that the Guinier approximation is only efective for small values of k.
   \begin{figure}[H]
      \label{fig:structure}
      \begin{minipage}{0.95\linewidth}
      \centering
      \includegraphics[height=4cm]{../code/structure_factor.png}
      \end{minipage}
      \caption{Structure factor (simulated)}
      \label{fig:structure}
  \end{figure}
   \begin{figure}[H]
      \label{fig:comparison}
      \begin{minipage}{0.95\linewidth}
      \centering
      \includegraphics[height=4cm]{../code/Guinier_approx.png}
      \end{minipage}
      \caption{Structure factor comparison (simulation and Guinier approximation)}
  \end{figure}
 \subsection{Polymer Extension}

 The Monte-Carlo simulation showed a good agreement with the theoretical results. The convergence of the extension, through the random iterations, is appreciable in figure \ref{fig:ex4_1}, the extension reaches the theoretical value near the step 2000. On the other hand, on \ref{fig:ex4_2} two curves are plotted, one is the curve for the relation of the force applied to the monomer to the mean value of the extension, is qualitatively appreciable that both curves touch and differ only by small values. 

  The agreement between the Monte Carlo simulation and the analytical Freely--Jointed Chain (FJC) force--extension relation was quantified using the root--mean--square error (RMSE),
  \begin{equation}
  \mathrm{RMSE} = \sqrt{\frac{1}{N_f}\sum_{i=1}^{N_f}
  \left(Q_x^{\mathrm{MC}}(F_i) - Q_x^{\mathrm{FJC}}(F_i)\right)^2},
  \end{equation}
  where $Q_x^{\mathrm{MC}}$ denotes the Monte Carlo estimate of the mean end--to--end extension along the force direction, $Q_x^{\mathrm{FJC}}$ is the theoretical prediction, and $N_f$ is the number of force values considered.

  For forces in the range $0$--$10~\mathrm{pN}$, the RMSE was found to be
  \[
  \mathrm{RMSE} = 2.19~\mathrm{nm},
  \qquad
  \frac{\mathrm{RMSE}}{Nb} = 2.2\times10^{-2},
  \]
  corresponding to an average deviation of approximately $2\%$ of the polymer contour length. This result indicates strong quantitative agreement between the Monte Carlo simulation and the FJC theoretical prediction, with deviations attributable to finite chain length and thermal fluctuations.


\begin{figure}[htbp]
    \centering

    \fbox{\includegraphics[width=\linewidth]{figures/step_1.png}}\par\vspace{0.5em}
    \fbox{\includegraphics[width=\linewidth]{figures/step_4.png}}\par\vspace{0.5em}
    \fbox{\includegraphics[width=\linewidth]{figures/step_60.png}}\par\vspace{0.5em}
    \fbox{\includegraphics[width=\linewidth]{figures/step_99.png}}

    \caption{Extension of the polymer using Monte-Carlo simulation}
    \label{fig:four-vertical-images}
\end{figure}
  
   \begin{figure}
    \begin{center}
      \includegraphics[width=0.45\textwidth]{figures/Q_x_step.pdf}
    \end{center}
    \caption{Extension evolution of the polymer, sturating to the theoretical value}\label{fig:ex4_1}
  \end{figure}
  
  \begin{figure}
    \begin{center}
      \includegraphics[width=0.45\textwidth]{figures/extension.pdf}
    \end{center}
    \caption{Extension - Force curve }\label{fig:ex4_2}
  \end{figure}
  
  
	
	\section{Conclusion}
	
The numerical simulations of the Freely Jointed Chain (FJC) model show excellent agreement with theoretical predictions across all investigated observables. The mean-square end-to-end distance $\langle Q^{2} \rangle$ and the mean-square radius of gyration $\langle R_{g}^{2} \rangle$ scale linearly with the chain length $N$, in accordance with the ideal-chain relations $\langle Q^{2} \rangle = Nb^{2}$ and $\langle R_{g}^{2} \rangle = Nb^{2}/6$. The probability distribution of the end-to-end distance $Q$ obtained from simulations closely follows the expected Gaussian form, confirming correct sampling of the equilibrium ensemble. The numerically computed structure factor $I(k)$ reproduces the characteristic low-$k$ behavior of an ideal polymer and agrees well with the Guinier approximation in the appropriate regime. Furthermore, Monte Carlo simulations of polymer stretching under an external force yield a force--extension curve in strong quantitative agreement with the analytical FJC prediction. The root--mean--square error between the simulated and theoretical mean extension in the force range $0$--$10\,\mathrm{pN}$ is $\mathrm{RMSE}=2.19\,\mathrm{nm}$, corresponding to approximately $2\%$ of the polymer contour length, with deviations attributable to finite chain length effects and thermal fluctuations.
	\begin{itemize}
		\item General conclusion about FJC model
		\item Summary of the main results
		\item Perspectives
	\end{itemize}
  \nocite{*}
\section{References}
\renewcommand{\refname}{}
\vspace{-7mm}
\bibliography{cite}
\bibliographystyle{plain}
	
	%%%FOOTNOTES%%%
	
\footnotetext{Dr. Adrien NICOLAÏ \& Pr. Patrick SENET}
	
\end{document}
