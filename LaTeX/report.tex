\documentclass[twoside,twocolumn,9pt]{article}
\usepackage{extsizes}
\usepackage[super,sort&compress,comma]{natbib} 
\usepackage[version=3]{mhchem}
\usepackage[left=1.5cm, right=1.5cm, top=1.785cm, bottom=2.0cm]{geometry}
\usepackage{balance}
\usepackage{times,mathptmx}
\usepackage{sectsty}
\usepackage{graphicx} 
\usepackage{lastpage}
\usepackage[format=plain,justification=justified,singlelinecheck=false,font={stretch=1.125,small,sf},labelfont=bf,labelsep=space]{caption}
\usepackage{float}
\usepackage{fancyhdr}
\usepackage{fnpos}
\usepackage[english]{babel}
\addto{\captionsenglish}{%
	\renewcommand{\refname}{Notes and references}
}
\usepackage{array}
\usepackage{droidsans}
\usepackage{charter}
\usepackage[T1]{fontenc}
\usepackage[dvipsnames]{xcolor}
\usepackage{setspace}
\usepackage[compact]{titlesec}
\usepackage{hyperref}
\usepackage{multirow}
\usepackage{tikz}
%%%Please don't disable any packages in the preamble, as this may cause the template to display incorrectly.%%%

%\usepackage{epstopdf}%This line makes .eps figures into .pdf - please comment out if not required.

\definecolor{cream}{RGB}{222,217,201}

\begin{document}
	
	\pagestyle{fancy}
	\thispagestyle{plain}
	\fancypagestyle{plain}{
		
		%%%HEADER%%%
		\renewcommand{\headrulewidth}{0pt}
	}
	%%%END OF HEADER%%%
	
	%%%PAGE SETUP - Please do not change any commands within this section%%%
	\makeFNbottom
	\makeatletter
	\renewcommand\LARGE{\@setfontsize\LARGE{15pt}{17}}
	\renewcommand\Large{\@setfontsize\Large{12pt}{14}}
	\renewcommand\large{\@setfontsize\large{10pt}{12}}
	\renewcommand\footnotesize{\@setfontsize\footnotesize{7pt}{10}}
	\makeatother
	
	\renewcommand{\thefootnote}{\fnsymbol{footnote}}
	\renewcommand\footnoterule{\vspace*{1pt}% 
		\color{cream}\hrule width 3.5in height 0.4pt \color{black}\vspace*{5pt}} 
	\setcounter{secnumdepth}{5}
	
	\makeatletter 
	\renewcommand\@biblabel[1]{#1}            
	\renewcommand\@makefntext[1]% 
	{\noindent\makebox[0pt][r]{\@thefnmark\,}#1}
	\makeatother 
	\renewcommand{\figurename}{\small{Fig.}~}
	\sectionfont{\sffamily\Large}
	\subsectionfont{\normalsize}
	\subsubsectionfont{\bf}
	\setstretch{1.125} %In particular, please do not alter this line.
	\setlength{\skip\footins}{0.8cm}
	\setlength{\footnotesep}{0.25cm}
	\setlength{\jot}{10pt}
	\titlespacing*{\section}{0pt}{4pt}{4pt}
	\titlespacing*{\subsection}{0pt}{15pt}{1pt}
	%%%END OF PAGE SETUP%%%
	
	%%%FOOTER%%%
	\fancyfoot{}
	\fancyhead{}
	\renewcommand{\headrulewidth}{0pt} 
	\renewcommand{\footrulewidth}{0pt}
	\setlength{\arrayrulewidth}{1pt}
	\setlength{\columnsep}{6.5mm}
	\setlength\bibsep{1pt}
	%%%END OF FOOTER%%%
	
	%%%FIGURE SETUP - please do not change any commands within this section%%%
	\makeatletter 
	\newlength{\figrulesep} 
	\setlength{\figrulesep}{0.5\textfloatsep} 
	
	\newcommand{\topfigrule}{\vspace*{-1pt}% 
		\noindent{\color{cream}\rule[-\figrulesep]{\columnwidth}{1.5pt}} }
	
	\newcommand{\botfigrule}{\vspace*{-2pt}% 
		\noindent{\color{cream}\rule[\figrulesep]{\columnwidth}{1.5pt}} }
	
	\newcommand{\dblfigrule}{\vspace*{-1pt}% 
		\noindent{\color{cream}\rule[-\figrulesep]{\textwidth}{1.5pt}} }
	
	\makeatother
	%%%END OF FIGURE SETUP%%%
	
	%%%TITLE AND AUTHORS%%%
	\twocolumn[
	\vspace{-1cm}
	\begin{@twocolumnfalse}	
		International Master 1 Physics, Photonics, Nanotechnology / Quanteem \hfill Soft Matter 2025-2026
		\begin{center}
			\noindent\LARGE{\textbf{Numerical Simulations of Ideal Chain Model of Polymer\\using the Freely Jointed Chain (FJC)}}\\
			\vspace{.5cm}
			\noindent\large{Andres Alvarez \& Diego Guerrero} \\
		\end{center}	
	\end{@twocolumnfalse} \vspace{0.6cm}
	
	]
	%%%END OF TITLE, AUTHORS AND ABSTRACT%%%
	
	%%%FONT SETUP - please do not change any commands within this section
	\renewcommand*\rmdefault{bch}\normalfont\upshape
	\rmfamily
	\section*{}
	\vspace{-1cm}
	
	%%%MAIN TEXT%%%
	
	\section{Introduction}
	
	\begin{itemize}
		\item General introduction about physics of polymer. Discrete models. Examples
		\item Theory of FJC model.
		\item Presentation of measure of extent (metrics). Relationship with experimental measurement.
		\item Goal of the practical work
	\end{itemize}
Polymer physics deals with complex substances through a simple approach. Understanding the properties of polymers from a molecular point of view is key to unveil their complexity.

Given an ensamble of polymers, we can use tools from statistics to derive some properties that apply to the ensamble at any given instant. To do so we model each polymer as a (continuous or discrete) ideal chain of monomers along with some constraints. Some examples of these models include the Freely Jointed Chain, the Freely Rotating Chain, the wormlike chain, the hindered rotating chain, the rotational isometric state model, among others \cite{Rubinstein}.


The first, most straightforward (yet illustrating)  approach is the Freely Jointed Chain model. The FJC is a chain consisting of \(N\) links, each of length \(b\) and able to point in any direction independently of each other (thus the simplicity).


	\section{Methods}
	
	\begin{itemize}
		\item Numerical simulations: Parameters + Brief description of the algorithm
		\item Examples of structures generated
		\item Time series + mean square and distributions statistical tools (theory) + Monte-Carlo
		\item Summary of data produced (ex: Table)
	\end{itemize}

  The first script analyzes the simulation data of a \(3D\) FJC polymer for different chain lenghts \(N\). 
  The parameters are the bond length \(b\), the \(N\) values we wish to test for, and the number of configurations \(T\) we wish to analyze.
  An XYZ file containing T polymer configurations is read, and for every configuration the end-to-end distance \(Q\) and the radius of gyration \(R_{g}\) is computed. Finally both quantities are squared and averaged over all configurations yielding \(\langle Q^2\rangle$ and $\langle R_g^2\rangle\). This algorithm is looped over for every \(N\).	


The second script is built over the first one. The only extra parameter needed are the number of bins for the histogram. After computing the end-to-end distance \(Q\), the algorithm builds a histogram out of these values, treating them as samples from a probability distribution. The histogram is then normalized so that it can be directly compared to the gaussian distribution.


The third simulation focuses on computing the structure factor for a polymer of \texttt{N = 100} and \texttt{b = 3}. An array \(k\) is introduced, spanning values from 0 to 1 (in our case, sampling every 0.01 steps). One of the files containing the simulated polymers is held in memory. Then, we use the formula  

\begin{equation}
I(k) = \sum_{i=0}^{N} \sum_{j=0}^{N}
\left\langle
\frac{\sin\!\left(k\,|\mathbf{R}_i - \mathbf{R}_j|\right)}
     {k\,|\mathbf{R}_i - \mathbf{R}_j|}
\right\rangle
\end{equation}
where \(\mathbf{R}_{i}\),\(\mathbf{R}_{i}\) are coordinates of the monomers in 3D space obtained from the simulated polymer.

Then, the Guinier approximation is computed on the same \(k\) array to compare them side by side.



	\begin{figure}[ht]
		\label{figure1}
		\begin{minipage}{.45\linewidth}
			\centering
			\includegraphics[height=4cm]{img1.jpg}
		\end{minipage}
		\hfill
		\begin{minipage}{.45\linewidth}
			\centering
			\includegraphics[height=4cm]{img2.jpg}
		\end{minipage}		
		\caption{Two polymer structures generated from FJC model simulations for $N=100$ and $b=3.0$.}
	\end{figure}
	
	\section{Results and Discussion}

	\begin{itemize}
		\item Description of the numerical results (Parts 1 to 4)
		\item Comparison with theoretical results (qualitatively and quantitatively)
		\item Discussion
	\end{itemize}
	\begin{figure}[ht]
		\begin{minipage}{.45\linewidth}
			\centering
			\includegraphics[height=4cm]{../code/Q2_vs_N_theory.pdf}
		\end{minipage}
		\hfill
		\begin{minipage}{.45\linewidth}
			\centering
			\includegraphics[height=4cm]{../code/Rg2_vs_N_theory.pdf}
		\end{minipage}		
		\caption{$\langle Q^{2}\rangle$ (left) and $\langle R_{g}^{2}\rangle$ (right)}
	\end{figure}

  \begin{figure}[H]
      \label{fig:histogram}
      \begin{minipage}{0.95\linewidth}
      \centering
      \includegraphics[height=4cm]{../code/Histogram_Q.pdf}
      \end{minipage}
      \caption{Probability distribution}
  \end{figure}
   \begin{figure}[H]
      \label{fig:structure}
      \begin{minipage}{0.95\linewidth}
      \centering
      \includegraphics[height=4cm]{../code/structure_factor.png}
      \end{minipage}
      \caption{Structure factor (simulated)}
  \end{figure}
   \begin{figure}[H]
      \label{fig:comparison}
      \begin{minipage}{0.95\linewidth}
      \centering
      \includegraphics[height=4cm]{../code/Guinier_approx.png}
      \end{minipage}
      \caption{Structure factor comparison (simulation and Guinier approximation)}
  \end{figure}
  
	
	\section{Conclusion}
	
	\begin{itemize}
		\item General conclusion about FJC model
		\item Summary of the main results
		\item Perspectives
	\end{itemize}
  \nocite{*}
\section{References}
\renewcommand{\refname}{}
\vspace{-7mm}
\bibliography{cite}
\bibliographystyle{plain}
	
	%%%FOOTNOTES%%%
	
\footnotetext{Dr. Adrien NICOLAÏ \& Pr. Patrick SENET}
	
\end{document}
